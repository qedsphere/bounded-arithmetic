Search up what bounded and sharply bounded quantifiers are, if you don't know.
\begin{definition}[Quantifier alternation classes]
    Note the following.
    \begin{itemize}
        \item $\Delta_0^b =\Sigma_0^b = \Pi_0^b$ is the set of sharply bounded formulas, and are P. 
        \item $\Sigma^b_{i+1}$ is the closure of $\Pi^b_{i}$ under $\wedge, \vee$, sharply bounded quantifiers, and bounded existential quantifiers.
        \item $\Pi^b_{i}$ is the closure of $\Sigma^b_{i+1}$ under $\wedge, \vee$, sharply bounded quantifiers, and bounded universal quantifiers.
    \end{itemize}
\end{definition}

\begin{definition}[$S^1_2$]
    BASIC is a set of axioms defining desired properties of our $\wedge, \vee, \#$, etc. P-induction is a set of axioms acting on statements 
    in a class of formulas. For $A\in \phi$, it says 
    $$A(0) \wedge (\forall x)(A(\lfloor \frac{1}{2}x \rfloor)\to A(x)) \to (\forall x) A(x)$$
    representing polynomially feasible induction. 

    Then $S^i_2$ is the set of axioms BASIC + $\Sigma^b_i$-PIND. 
\end{definition} 

\begin{theorem}[Main Theorem for $S^1_2$]
    The set of $\Sigma^b_1$ definable functions provable from $S^1_2$ is the set of problems in $P$.
\end{theorem}
\begin{proof}
    This is proven using sequent calculus and witnessing lemma. It is summarized \href{https://proof2025.ugent.be/talks/summer_school/tutorial-RJalali_slides_pt2.pdf}{here}.
    We will discuss such proofs soon.
\end{proof}

First, we return to a motivating (open) question:
\begin{problem}
    Is full bounded arithmetic finitely axiomatizable? In particular, is $S^1_2 = S_2$? (Where $S_2$ is defined as the union of all $S^i_2$'s.)
\end{problem}

\begin{theorem}
$BT=S^1_2 + dWPHP(\Delta^b_1)$ is finitely axiomatized by the instance of $\dWPHP$ on the circuit value function $\CV(x, y)$. 
\end{theorem}

It is good to know the following principle $\dWPHP_1(f, g)$:

\begin{definition}[$\dWPHP_1(f, g)$]
    $\exists y<2a \; g(y)\ge a\vee f(g(y))\ne y$
\end{definition}

Skip forward a bit in PCG, and we have the following conservativity question relationships: 

\begin{center}
$S_2^1(P V) \preceq_{\Sigma_1^b} B T \Leftrightarrow P V_1 \preceq_{\Sigma_1^b} B T \Leftrightarrow P V_1 \preceq_{\Sigma_1^b} A P C_1$
\end{center}

and the following equivalencies:
\begin{itemize}
    \item $S^1_2\ne BT$ iff $S^1_2$ does not prove $\dWPHP(CV)$;
    \item $S^1_2\preceq_{\Sigma^b_1} BT$ iff $S^1_2$ does not prove $\dWPHP_1(CV, CV)$.
\end{itemize}

Furthermore, within $PV$, we can always find a p-time function $g$ without parameters such that $S^1_2(\PV)$ proves $\dWPHP(g)\to \dWPHP(f)$ for all p-time $f$.
Namely $g$ is the truth-table function. However it is not known whether this is true for $S^1_2(\PV_1)$. 